\documentclass[12pt]{extarticle}
\usepackage[margin=3cm, footskip=55pt]{geometry}
\usepackage{fancyhdr}
\usepackage{titling}
\usepackage{filecontents}
\usepackage{multicol}

\setlength{\droptitle}{-10em} 
\pagestyle{empty}
\pagestyle{fancy}
\fancyhf{}
\renewcommand{\headrulewidth}{0pt}
\lfoot{CMPT 330 Lab 1}
\rfoot{Page: \thepage}
\title{CMPT 330 Lab1}
\author{Tyler Kuipers}
\date{September 8, 2015}
\fancypagestyle{plain}{%
	\renewcommand{\headrulewidth}{0pt}%
	\fancyhf{}%
	\lfoot{CMPT360 Lab1}
	\rfoot{Page: \thepage}
	% \fancyfoot[C]{\footnotesize Page \thepage\ of \pageref{LastPage}}%
}

\begin{document}
	\maketitle
	\section*{Lab1A Questions}
		\begin{enumerate}
			\item Mystery.c ran a fork bomb on the computer.  A fork bomb is a program with the sole purpose of spawning more versions of itself.  It does this enough times that eventually, the computer will not function properly.
			\item The man command looks up the manual for the given command.  It then prints it to the console for the user to look at using a simple interface.
			\item Man Page Sections\begin{enumerate}
				\item 1. Commands
				\item 2. System Calls
				\item 3. C Library Calls
				\item 4. Special files (Devices)
				\item 5. File Formats and Conventions
				\item 6. Games
				\item 7. Conventions and Miscellaneous
				\item 8. System Management Commands
			\end{enumerate}
			\item Fork creates a child process.
			\item Fork resides in section 2 and 3. In section 2, the manual says that a child process is created.  In section three, the manual says that a new process is created.
			\item Unistd.h defines miscellaneous symbolic constants and types and declares miscellaneous functions.
			\item Unistd.h resides in section 7.
			\item CHILD\_MAX is 7350.
		\end{enumerate}
	\clearpage

	\section*{Lab1B Questions}
		\begin{enumerate}
			\item List Items with Sections
			\begin{multicols}{3}
				 \begin{enumerate}fork 2 3posix 
					\item[] waitpid 2 3posix 
					\item[] execve 2 3posix 
					\item[] exit 1posix 2 3 3posix 
					\item[] open 1 2 3posix 
					\item[] close 2 3posix 
					\item[] read 1posix 2 3posix 
					\item[] write 1 1posix 2 3posix 
					\item[] lseek 2 3posix 
					\item[] stat 1 2 3posix 
					\item[] mkdir 1 1posix 2 3posix 
					\item[] rmdir 1 1posix 2 3posix 
					\item[] link 1 1posix 2 3posix 
					\item[] ln 1 1posix 
					\item[] unlink 1 1posix 2 3posix 
					\item[] mount 2 8 
					\item[] umount 2 8 
					\item[] chdir 2 3posix 
					\item[] fflush 3 3posix 
					\item[] time 1posix 2 3posix 7 
					\item[] ps2pdf 1 
					\item[] man 1 1posix 7 
					\item[] threads 3ssl 
					\item[] kill 1 1posix 2 3posix 
					\item[] mkdir 1 1posix 2 3posix 
					\item[] chmod 1 1posix 2 3posix 
					\item[] chown 1 1posix 2 3posix 
					\item[] stdio.h 7posix 
					\item[] stdlib.h 7posix 
					\item[] mv 1 1posix 
					\item[] rm 1 1posix 
					\item[] cp 1 1posix 
					\item[] bash 1 
					\item[] sh 1 1posix 
					\item[] zsh 1 
					\item[] env 1 1posix 
					\item[] lpr 1 
					\item[] intro 1 2 3 4 5 6 7 8 
					\item[] wc 1 1posix 
				\end{enumerate}
			\end{multicols}
			\noindent\makebox[\linewidth]{\rule{\paperwidth}{0.4pt}}
			There are some from the posix guide and the default guide.  This is why things like mkdir can be in manual \#1 twice.  A simpler list is as follows:\\
			\begin{multicols}{3}
				\begin{enumerate}
					\item[] fork 2 3 
					\item[] waitpid 2 3 
					\item[] execve 2 3 
					\item[] exit 1 2 3 3 
					\item[] open 1 2 3 
					\item[] close 2 3 
					\item[] read 1 2 3 
					\item[] write 1 1 2 3 
					\item[] lseek 2 3 
					\item[] stat 1 2 3 
					\item[] mkdir 1 1 2 3 
					\item[] rmdir 1 1 2 3 
					\item[] link 1 1 2 3 
					\item[] ln 1 1 
					\item[] unlink 1 1 2 3 
					\item[] mount 2 8 
					\item[] umount 2 8 
					\item[] chdir 2 3 
					\item[] fflush 3 3 
					\item[] time 1 2 3 7 
					\item[] ps2pdf 1 
					\item[] man 1 1 7 
					\item[] threads 3 
					\item[] kill 1 1 2 3 
					\item[] mkdir 1 1 2 3 
					\item[] chmod 1 1 2 3 
					\item[] chown 1 1 2 3 
					\item[] stdio.h 7 
					\item[] stdlib.h 7 
					\item[] mv 1 1 
					\item[] rm 1 1 
					\item[] cp 1 1 
					\item[] bash 1 
					\item[] sh 1 1 
					\item[] zsh 1 
					\item[] env 1 1 
					\item[] lpr 1 
					\item[] intro 1 2 3 4 5 6 7 8 
					\item[] wc 1 1
				\end{enumerate}
			\end{multicols}
			\item Amount in Each Section
			\begin{multicols}{3}
				\begin{enumerate}
					\item 1. 27
					\item 2. 23
					\item 3. 23
					\item 4. 1
					\item 5. 5
					\item 6. 1
					\item 7. 5
					\item 8. 3
				\end{enumerate}
			\end{multicols}
			\item \textit{\textdollar\textgreater lpr rmdir\textbackslash(1\textbackslash).pdf} tries to print the document `rmdir(1).pdf' file.  The backslashes are in this command because ( and ) are special characters in bash scripting that need to be escaped.
			\item Shell scripting is extremely useful when you need to automate something that would take a lot of command line work.  For example, when I wipe my hard drive, it would be convenient to have an install script.  This script would, once I have installed my entire operating system, go through and set up my development environment by installing all of the programs I need, and grabbing configuration files from my GitHub.
		\end{enumerate}
\end{document}