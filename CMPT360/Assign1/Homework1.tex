\documentclass[12pt]{extarticle}
\usepackage[margin=3cm, footskip=55pt]{geometry}
\usepackage{fancyhdr}
\usepackage{titling}
\setlength{\droptitle}{-10em} 
\pagestyle{empty}
\pagestyle{fancy}
\fancyhf{}
\renewcommand{\headrulewidth}{0pt}
\lfoot{Math 360 Assign 1}
\rfoot{Page: \thepage}
\title{CMPT 360 Assign1 Section 1\\ Question's 1, 4, 5, 6, 7, 13, 19, 21}
\author{Tyler Kuipers}
\date{September 8, 2015}
\fancypagestyle{plain}{%
	\renewcommand{\headrulewidth}{0pt}%
	\fancyhf{}%
	\lfoot{Math 360 Assign 1}
	\rfoot{Page: \thepage}
	% \fancyfoot[C]{\footnotesize Page \thepage\ of \pageref{LastPage}}%
}

\begin{document}
\maketitle
\section*{Question 1}
	Which of the following are propositions? 
	What are the truth values of the propositions?
	\begin{enumerate}
		\item True
		\item False
		\item True
		\item False
		\item Not a proposition, $x$ is not fixed.
		\item Not a proposition, only a command.
		\item True, Is a proposition because both $x$ and $y$ are given testable sets of values.
	\end{enumerate}

\section*{Question 4}
	Let $p$ and $q$ be propositions:\\
		\hspace*{1cm}$p$: I bought a lottery ticket this week.\\
		\hspace*{1cm}$q$: I won the million dollar jackpot on Friday.\\
	\begin{enumerate}
		\item I did not buy a lottery ticket this week.
		\item I bought a lottery ticket this week or I won the lottery
		\item If I buy a lottery ticket this week, I will win the million dollar jackpot on Friday.
		\item I bought a lottery ticket this week and I won the million dollar jackpot on Friday
		\item I can only win the lottery on Friday if I buy a ticket this week.
		\item If I don't buy a ticket this week, I won't win the lottery on Friday
		\item I didn't buy a lottery ticket this week, and I didn't win it on Friday
		\item Either I buy a lottery ticket this week and win a million dollars, or I don't buy a lottery ticket. (or both)
	\end{enumerate}

\section*{Question 5}
	Let $p$ and $q$ be propositions:\\
		\hspace*{1cm}$p$: It is below freezing.\\
		\hspace*{1cm}$q$: It is snowing\\
	\begin{enumerate}
		\item $p \wedge q$
		\item $p \wedge \neg q$
		\item $\neg p \wedge \neg q$
		\item $p \vee q$
		\item $p \to q$
		\item $(p \vee q) \wedge (p \to \neg q)$
		\item $p \leftrightarrow q$
	\end{enumerate}

\section*{Question 6}
	Let $p$, $q$, and $r$ be propositions:\\
		\hspace*{1cm}$p$: You have the flu.\\
		\hspace*{1cm}$q$: You miss the final exam.\\
		\hspace*{1cm}$r$: You pass the course.\\
	\begin{enumerate}
		\item You have the flu so you miss the final.
		\item You don't miss the final exam and because of this, you can't fail the course.
		\item You miss the final so you fail the course.
		\item You either have the flu, miss the final, or fail the course. (or any combination of those three)
		\item You either have the flu and fail or miss the final and fail. (or both) 
		\item You either have the flu and miss the final, or don't have the flu and pass. (or both)
	\end{enumerate}
	\clearpage

\section*{Question 7}
	Let $p$ and $q$ be propositions:\\
		\hspace*{1cm}$p$: You drive over 65 mph.\\
		\hspace*{1cm}$q$: You get a speeding ticket.\\
	\begin{enumerate}
		\item $\neg p$
		\item $p \wedge \neg q$
		\item $p \to q$
		\item $\neg p \to \neg q$
		\item $p \to q$
		\item $\neg p \wedge q$
		\item $q \to p$
	\end{enumerate}
\clearpage
\section*{Question 13}
	Each inhabitant of a remote village always tells the truth or always lies. A villager will only give a `Yes' or a `No' response to a question a tourist asks. Suppose you are a tourist visiting this area and come to a fork in the road. One branch leads to the ruins you want to visit; the other branch leads deep into the jungle. A villager is standing at the fork in the road. What one question can you ask the villager to determine which branch to take?\\\\
	I would ask `If I were to ask you whether the path to the right leads to the ruins, would you say yes?'\\\\
	Let:\\
		\hspace*{1cm}$p$ represent whether or not the villager is a liar.\\
		\hspace*{1cm}$q$ represent whether or not the branch to the right leads to the ruins. (hidden information).\\
		\hspace*{1cm}$r$ represent the villager's answer to the question `Does the path to the right lead to the ruins?'.\\
		\hspace*{1cm}$s$ represent the villager's answer to the question `What would you say if I asked whether the path to the right leads to the ruins?'\\
	\begin{center}
		\begin{tabular}{ | p{2cm} | p{2cm} | p{5cm} | p{5cm} |}
			\hline$p$ & $q$ & $r = (p \leftrightarrow \neg q) \oplus (\neg p \leftrightarrow q)$ & $s = (p \leftrightarrow \neg r) \oplus (\neg p \leftrightarrow  r)$ \\ \hline \hline
			$False$ & $False$ & $False$ & $False$\\ \hline 
			$False$ & $True$ & $True$ & $True$\\ \hline
			$True$ & $False$ & $True$ & $False$\\ \hline
			$True$ & $True$ & $False$ & $True$\\ \hline
		\end{tabular}
	\end{center}
	In the above table, we show that if the villager is a liar, and will always lie, we can use that to determine the path.  Since we expect him to lie, we ask whether his answer would be yes to taking the path on the right.  This lets us know the truth because if he were lying about whether the path on the right leads to the ruins, he will lie again and they will cancel out.  If the villager is not a liar, then we have nothing to worry about.  Because of this, if we ask `If I were to ask you whether the path to the right leads to the ruins, would you say yes?', and the villager says yes, we can be confident that the path to the right is correct.

\section*{Question 19}
	State the converse and contrapositive of each of the following implications.
	\begin{enumerate}
		\item Converse: If I ski tomorrow, it will snow today.\\
			  Contrapositive: If it doesn't snow today, I won't ski tomorrow.
		\item Converse: Whenever I come to class, there will be a quiz.\\
			  Contrapositive: If there is not a quiz, I won't come to class.
		\item Converse: If a positive integer has no divisors other than one and itself, it is a prime number.\\
			  Contrapositive: If a positive integer has divisors other than one and itself, it is not prime.
	\end{enumerate}

\section*{Question 21}
	Construct a truth table for each of the following compound prepositions.
	\begin{enumerate}
		\item $p \wedge \neg p$\\
			\begin{tabular}{ | p{2cm} | p{2cm} |}
				\hline $p$ & $p \wedge \neg p$\\ \hline\hline
				$0$ & $0$ \\ \hline
				$1$ & $0$ \\ \hline 
			\end{tabular}


		\item $p \vee \neg p$\\
			\begin{tabular}{ | p{2cm} | p{2cm} |}
				\hline $p$ & $p \vee \neg p$\\ \hline\hline
				$0$ & $1$ \\ \hline
				$1$ & $1$ \\ \hline 
			\end{tabular}

		\item $(p \vee \neg q) \to q$\\
			\begin{tabular}{ | p{2cm} | p{2cm} | p{2cm} | p{2cm} | p{3cm} |}
				\hline $p$ & $q$ & $\neg q$ & $p \vee \neg q$ & $(p \vee \neg q) \to q$\\ \hline\hline
				$0$ & $0$ & $1$ & $1$ & $0$\\ \hline
				$0$ & $1$ & $0$ & $0$ & $1$\\ \hline
				$1$ & $0$ & $1$ & $1$ & $0$\\ \hline
				$1$ & $1$ & $0$ & $1$ & $1$\\ \hline
			\end{tabular}

		\item $(p \vee \neg q) \to q$\\
			\begin{tabular}{ | p{2cm} | p{2cm} | p{2cm} | p{2cm} | p{3cm} |}
				\hline $p$ & $q$ & $p \to q$ & $p \wedge q$ & $(p \vee q) \to (p \wedge q)$\\ \hline\hline
				$0$ & $0$ & $1$ & $1$ & $1$\\ \hline
				$0$ & $1$ & $1$ & $1$ & $0$\\ \hline
				$1$ & $0$ & $0$ & $0$ & $0$\\ \hline
				$1$ & $1$ & $1$ & $1$ & $1$\\ \hline
			\end{tabular}

		\item $(p \to q) \leftrightarrow (\neg q \to \neg p)$\\
			\begin{tabular}{ | p{2cm} | p{2cm} | p{2cm} | p{2cm} | p{3cm} |}
				\hline $p$ & $q$ & $p \to q$ & $\neg q \to \neg p$ & $(p \to q) \leftrightarrow (\neg q \to \neg p)$\\ \hline\hline
				$0$ & $0$ & $1$ & $1$ & $1$\\ \hline
				$0$ & $1$ & $1$ & $1$ & $1$\\ \hline
				$1$ & $0$ & $0$ & $0$ & $1$\\ \hline
				$1$ & $1$ & $1$ & $1$ & $1$\\ \hline
			\end{tabular}

		\item $(p \to q) \leftrightarrow (q \to p)$\\
			\begin{tabular}{ | p{2cm} | p{2cm} | p{2cm} | p{2cm} | p{3cm} |}
				\hline $p$ & $q$ & $p \to q$ & $q \to  p$ & $(p \to q) \leftrightarrow (q \to p)$\\ \hline\hline
				$0$ & $0$ & $1$ & $1$ & $1$\\ \hline
				$0$ & $1$ & $1$ & $0$ & $0$\\ \hline
				$1$ & $0$ & $0$ & $1$ & $1$\\ \hline
				$1$ & $1$ & $1$ & $1$ & $1$\\ \hline
			\end{tabular}

	\end{enumerate}	
	

\end{document}